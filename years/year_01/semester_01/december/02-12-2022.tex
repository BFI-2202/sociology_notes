\documentclass{article}
\usepackage[utf8]{inputenc}

\usepackage[T2A]{fontenc}
\usepackage[utf8]{inputenc}
\usepackage[russian]{babel}

\title{Социология}
\author{Лисид Лаконский}
\date{December, 2022}

\begin{document}

\maketitle
\tableofcontents
\pagebreak

\section{Социология - 02.12.2022}

\subsection{Социализация личности}

\begin{flushleft}

\textbf{Период} социализации - от рождения человека до его смерти (условие - если он живет в социуме).

\textbf{Социализация} - закрепление социальных правил, традиционных в данном обществе. (неформальное определение)

К \textbf{потере} основных социальных черт личности приводят различные зависимости и так далее - \textbf{деградация личности}.

\hfill

В социальной психологии выделяют несколько \textbf{граней личности}: \textbf{телесную} (физическое я), \textbf{социально-ролевую} (осознание себя в качестве носителя социальных ролей), \textbf{психологическое я} (мотивы, внутренние стимулы, восприятие собственных особенностей), \textbf{ощущение себя как источника активности или пассивности}, \textbf{самоотношение (смысловое я;} самооценка)

\textbf{Социализация} - процесс и результат включения индивида в социальные отношения; социализация \textbf{осуществляется} путем \textbf{усвоения социального опыта} и \textbf{воспроизведения его в социальной деятельности}. 

\hfill

Социализация \textbf{осуществляется через ряд условий}:

\begin{enumerate}
    \item Целенаправленные условия
    \begin{enumerate}
        \item Зона ближайшего развития (Лев Семенович Выготский) - семья (мама, папа, родные братья и сестры)
        \item Образовательные учреждения (детский сад, школа, высшие учебные заведения) посредством педагогического процесса
        \item Рабочая деятельность
    \end{enumerate}
    \item Случайные воздействия - произведения искусства, воспринимаемые нами; случайные события
\end{enumerate}

\textbf{Механизмы} социализации:

\begin{enumerate}
    \item Идентификация - \textbf{отождествление} индивида с некоторыми людьми, что позволяет усваивать разнообразные \textbf{отношения, формы поведения}, которые характерны окружающим.
    \item Подражание - сознательное или бессознательное воспроизведение индивидом модели поведения, опыта других людей: манеры поведения, движения, поступки. Выходит за пределы зоны ближайшего развития.
    \item Внушение - процесс неосознанного повторения индивидуом внутреннего опыта, мыслей, состояния.
    \item Социальная фасилитация (облегчение) - стимулированное влияние на поведение одних людей на других, в результате которого их деятельность протекает более легко.
    \item Конформизм - осознание расхождения во мнениях с окружающими людьми, но внешнее согласие с ними, реализуемое в поведении.
\end{enumerate}

\pagebreak
\subsection{Теория личности}

\subsubsection{Восемь стадий развития личности по Эрику Эриксону}

\begin{minipage}[t]{0.48\textwidth}
\textbf{Нормальная линия развития}

\bigskip

\textbf{1) } Грудной возраст - доверие к людям

\textbf{2) } Автономия - самостоятельность (1-2 года)

\textbf{3) } Инициатива и активность (3-5 лет)

\textbf{4) } Младший школьный возраст - трудолюбие

\textbf{5) } Юность - жизненное самоопределение

\textbf{6) } Начало взрослого периода - близость к людям

\textbf{7) } Зрелый возраст - творческая активность (до 60 лет)

\textbf{8) } Доживание - полнота жизни (до смерти)
\end{minipage}
\begin{minipage}[t]{0.48\textwidth}
\textbf{Аномальная линия развития}

\bigskip

\textbf{1) } Грудной возраст - недоверие к людям

\textbf{2) } Автономия - сомнение (1-2 года)

\textbf{3) } Пассивность (3-5 лет)

\textbf{4) } Младший школьный возраст - чувство собственной неполноценности

\textbf{5) } Юность - путаница ролей

\textbf{6) } Начало взрослого периода - изоляция от людей

\textbf{7) } Зрелый возраст - застой (до 60 лет)

\textbf{8) } Доживание - отчание, стремление прожить жизнь с начала (до смерти)
\end{minipage}

\end{flushleft}

\end{document}
