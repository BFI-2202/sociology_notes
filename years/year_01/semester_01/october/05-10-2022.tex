\documentclass{article}
\usepackage[utf8]{inputenc}

\usepackage[T2A]{fontenc}
\usepackage[utf8]{inputenc}
\usepackage[russian]{babel}

\title{Социология}
\author{Лисид Лаконский}
\date{October, 2022}

\begin{document}

\maketitle
\tableofcontents
\pagebreak

\section{Социология - 05.10.2022}

Первая экономическая формация - отсутствие науки и научного знания - общинная.

Вторая экономическая формация - рабовладельческая экономическая формация - появление первых государств, первых систем научного познания: математики, физики, астрономии и так далее. Появление философии. Донаучный этап развития социологии, физики, много всего...

Донаучный этап развития социологии начинается с второй экономической формации.

Третья экономическая формация - феодальная экономическая формация, совпадает с эпохой средневековья. Разделяют раннее средневековье (13-15 века) и позднее средневековье. Карл Маркс и Фридрих Энегльс не признавали третью экономическую формацию. Большое влияние религии на науку, не дает развиваться полностью.

Ренессанс - отход от религиозной системы мировоззрения, появление первых светских университетов в Италии.

Четвертая экономическая формация - капиталистическая экономическая формация, эпоха нового времени. Окончательное разделение науки и религии, появление антропологии. Девятнадцатый век - господство естествознания, большое влияние на педагогику, социологию, психиатрию и так далее - новые науки. Их инструментарий включает в себя методы естественонаучного познания: эксперимент, наблюдение, измерение и так далее - все подтверждается на практике. Отличие от более ранних наук - например, истории, там другие методы.

Тридцатые годы девятнадцатого века - ключевая дата - социология становится самостоятельной отраслью научного познания. Научный этап развития социологии. Основоположник социологии - Огюст Конт.

\subsection{Развитие отечественной социологии}

После революции стали появляться кафедры социологии - подготовка специалистов - социология как отдельный социальный институт.

До революции ничего такого не было.

\subsection{Вопросы на контрольную работу}

Ответить на них необходимо на бумажке ручкой по листочку. Необходимо прислать сегодня (05.09.2022) вечером.

Где можно посмотреть - любой учебник по истории философии. Хороший учебник - минский.

Социология в России под редакцией Ядова. Прочитать параграф ,Развитие социологии в нашей стране'.

Темы докладов (кто не взял - просто читает):

\begin{enumerate}
    \item Социологическая концепция Мечникова.
    \item Социологическая концепция Михайловского.
    \item Социологическая концепция Лаврова.
    \item Общая характеристика развития отечественной социологии до 17-20 века.
    \item Развитие советской социологии.
\end{enumerate}

Ответ письменно на следующий вопрос:

\begin{enumerate}
    \item Донаучный этап развития социологии (через призму основных экономических формаций, Карл Маркс и Фридрих Энгельс)
\end{enumerate}

\end{document}