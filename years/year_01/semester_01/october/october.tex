\documentclass{article}
\usepackage[utf8]{inputenc}

\usepackage[T2A]{fontenc}
\usepackage[utf8]{inputenc}
\usepackage[russian]{babel}

\title{Социология}
\author{Лисид Лаконский}
\date{October, 2022}

\begin{document}

\maketitle
\tableofcontents
\pagebreak

\section{Социология - 05.10.2022}

Первая экономическая формация - отсутствие науки и научного знания - общинная.

Вторая экономическая формация - рабовладельческая экономическая формация - появление первых государств, первых систем научного познания: математики, физики, астрономии и так далее. Появление философии. Донаучный этап развития социологии, физики, много всего...

Донаучный этап развития социологии начинается с второй экономической формации.

Третья экономическая формация - феодальная экономическая формация, совпадает с эпохой средневековья. Разделяют раннее средневековье (13-15 века) и позднее средневековье. Карл Маркс и Фридрих Энегльс не признавали третью экономическую формацию. Большое влияние религии на науку, не дает развиваться полностью.

Ренессанс - отход от религиозной системы мировоззрения, появление первых светских университетов в Италии.

Четвертая экономическая формация - капиталистическая экономическая формация, эпоха нового времени. Окончательное разделение науки и религии, появление антропологии. Девятнадцатый век - господство естествознания, большое влияние на педагогику, социологию, психиатрию и так далее - новые науки. Их инструментарий включает в себя методы естественонаучного познания: эксперимент, наблюдение, измерение и так далее - все подтверждается на практике. Отличие от более ранних наук - например, истории, там другие методы.

Тридцатые годы девятнадцатого века - ключевая дата - социология становится самостоятельной отраслью научного познания. Научный этап развития социологии. Основоположник социологии - Огюст Конт.

\subsection{Развитие отечественной социологии}

После революции стали появляться кафедры социологии - подготовка специалистов - социология как отдельный социальный институт.

До революции ничего такого не было.

\subsection{Вопросы на контрольную работу}

Ответить на них необходимо на бумажке ручкой по листочку. Необходимо прислать сегодня (05.09.2022) вечером.

Где можно посмотреть - любой учебник по истории философии. Хороший учебник - минский.

Социология в России под редакцией Ядова. Прочитать параграф ,Развитие социологии в нашей стране'.

Темы докладов (кто не взял - просто читает):

\begin{enumerate}
    \item Социологическая концепция Мечникова.
    \item Социологическая концепция Михайловского.
    \item Социологическая концепция Лаврова.
    \item Общая характеристика развития отечественной социологии до 17-20 века.
    \item Развитие советской социологии.
\end{enumerate}

Ответ письменно на следующий вопрос:

\begin{enumerate}
    \item Донаучный этап развития социологии (через призму основных экономических формаций, Карл Маркс и Фридрих Энгельс)
\end{enumerate}

\pagebreak
\section{Социология - 07.10.2022}

Основные произведения Огюста Конта: курс позитивной философии, рассуждения о позитивной философии, система позитивной политики (трактат, устанавливающий новую религию человечества)

Позитивизм - междисциплинарное научное течение, основоположник - Огюст Конт.

\subsection{Первая форма позитивизма: классическая}

Позитивизм - это социально-философское направление, основанное на том, что все истинное подлинное научное знание может быть получено как результат отдельных специфических наук или их синтетического объединения.

Основоположником позитивизма является Огюст Конт.

Ярчайшим представителем позитивизма является Герберт Спенсер (основоположник английской социологии), Литере, Вырубок.

В основании классической формы позитивизма лежит Контовская концепция трех стадий (является также основой социальной динамики). Он считал, что все общества последовательно проходят три основные стадии:

\begin{enumerate}
    \item Теологическая стадия. Дифференцируется на три основные стадии:
    \begin{enumerate}
        \item Фетишизм - когда люди приписывают жизнь внешним предметам и видят в них богов.
        \item Политеизм - многобожие. Представители: Древняя Греция, Древний Рим.
        \item Монотеизм - единобожие.
    \end{enumerate}
    \item Метафизическая стадия - переходная стадия. Пример: эпоха ренессанса (возрождения) в Европе. Религия теряет свои позиции, наука становится самостоятельней. К этой стадии могут относиться революции.
    \item Позитивная стадия. Свидетельством о том, что общество вступает в позитивную эру, является распространение наук и рост их общественного значения, а также создание теории позитивизма.
\end{enumerate}

В конце двадцатого века - кризис классической формы позитивизма. Переосмысление основ позитивизма

\subsection{Вторая форма позитивизма: махизм (эмпириокритицизм)}

Махизм - это субъективное идеалистическое течение, основанное Эрнстом Махом и Рихардом Авенариусом; форма позитивизма конца девятнадцатого - начала двадцатого века.

Исходный пункт эмпириокритицизма - ,чистый опыт', истолкованный как последняя, нейтральная, не физическая и не психическая реальность.

Мир в эмпириокритицизме выступает как совокупность ощущений, к которой непреложны такие философские категории как материя, причина, дух.

Реальным является ощущение; то, что нельзя ощутить, нереально.

Тем, кто будет делать доклады - изучить биографию Эрнста Маха и Рихарда Авенариуса.

\subsection{Третья форма позитивизма: неопозитивизм}

Неопозитивизм существовал как международное интернациональное течение и зародился в объединении ученых различных специальностей в Венском кружке.

В Англии идеи непозитивизма развивал английский математик Бертран Рассел.

Неопозитивизм - это научная ориентация, которая опирается на принципы логического позитивизма.

Основные принципы неопозитивизма:

\begin{enumerate}
    \item Рассмотрение социальных являений на основе законов общих как для природной, так и для социально-экономической реальности - метод сравнения (аналогия).
    \item Использование методов естествознания в исследовании - традиция от Огюста Конта - именно он впервые обозначил, что социология как наука должна опираться на методы естественнонаучного познания.
    \item Операционализм - преобразование теоретического суждения с целью его эмпирической проверки. Данная тенденция прослеживается в естественных науках.
    \item Исследование субъективных факторов через поведение - бихевиоризм. Стимул - реакция. Бихевиористы исследовали, какие стимулы поступают в голову человека и какие реакции последуют на эти стимулы. Следующее течение - необихевиоризм - исследование ,черного ящика' в голове у человека: какие мысли (другие факторы) влияют на центр принятия решений в голове у человека, отчего получаются разные реакции на один и тот же символ. Бихевиоризм был бы невозможен в своем развитии без трудов Павлова и Бехтерева - заложили основы бихевиоризма. Основоположником бихевиоризма в социологии является американский ученый Беррес Фридерик Скиннер.
\end{enumerate}

Представители данного направления считали себя философами.

Следующая тема - социологическая концепция Огюста Конта - следующее лекционное занятие.

\end{document}